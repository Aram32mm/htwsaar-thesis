% ****************************************************************************************************
% htwsaar-i-mst-config.tex 
% ****************************************************************************************************  
\RequirePackage[utf8]{inputenc}
\DeclareUnicodeCharacter{00A0}{~}
\RequirePackage[T1]{fontenc}

% ********************************************************************
% Line spacing
% ********************************************************************
\usepackage{setspace} % allows single, one-and-a-half, or double spacing

% ********************************************************************
% TikZ and libraries for figures
% ********************************************************************
\usepackage{tikz}
\usetikzlibrary{positioning}

% ****************************************************************************************************
% 1. Personal data and user ad-hoc commands
% ****************************************************************************************************
\newcommand{\myTitle}{Retrieval-Augmented Generation for the Discovery of Payment Validation Rules via LLM-Enriched Source Data}
\newcommand{\myDegree}{Bachelor of Science (B.\,Sc.)\xspace}
%\newcommand{\myDegree}{Master of Science (M.\,Sc.)\xspace}
\newcommand{\myDegreeType}{Bachelor\xspace}
%\newcommand{\myDegreeType}{Master\xspace}
\newcommand{\myDegreeCourse}{Praktische Informatik}
%\newcommand{\myDegreeCourse}{Informatik und Web-Engineering}
%\newcommand{\myDegreeCourse}{Kommunikationsinformatik}
%\newcommand{\myDegreeCourse}{Produktionsinformatik}
\newcommand{\myName}{José Aram Méndez Gómez\xspace}
\newcommand{\myUni}{Hochschule für Technik und Wirtschaft des Saarlandes\xspace}
\newcommand{\myCompany}{Deutsche Bank AG\xspace}
\newcommand{\myFirstProf}{Prof. Dr. Markus Esch\xspace}
\newcommand{\mySecondProf}{Alexander-A Efremov\xspace}
\newcommand{\myLocation}{Saarbrücken\xspace}
\newcommand{\myTime}{August 2025\xspace}
\newcommand{\currentVersion}{Version 1.0, August 2025\xspace} % TODO: ggf. über git Versionsinformationen automatisch bereitstellen und verwenden

% ********************************************************************
% Setup, finetuning, and useful commands
% ********************************************************************
\newcounter{dummy} % necessary for correct hyperlinks (to index, bib, etc.)
% ****************************************************************************************************


% ****************************************************************************************************
% 2. Loading some handy packages
% ****************************************************************************************************
% ******************************************************************** 
% Packages with options that might require adjustments
% ******************************************************************** 
\PassOptionsToPackage{american}{babel}   % change this to your language(s)
 \RequirePackage{babel}					
 \RequirePackage{csquotes}
  
\PassOptionsToPackage{language=auto,style=numeric-comp,backend=biber,bibencoding=utf8,maxbibnames=50}{biblatex} % backend ggf. auf neueres Werkzeug (z.B. biber) anpassen
 \RequirePackage{biblatex}	
 %\bibliography{Bibliography}	% alter Befehl
 \addbibresource{Bibliography.bib}

\PassOptionsToPackage{fleqn}{amsmath}		% math environments and more by the AMS 
 \RequirePackage{amsmath}

% ******************************************************************** 
% Setting up the page and margins
% ******************************************************************** 
\usepackage{geometry}
 \geometry{a4paper,left=25mm,right=35mm,top=25mm,bottom=30mm}
% DIESE WERTE SIND NICHT ZU VERÄNDERN -- DO NOT CHANGE THESE VALUES

% ******************************************************************** 
% General useful packages
% ******************************************************************** 
%\usepackage[automark]{scrpage2}
\PassOptionsToPackage{dvipsnames}{xcolor}
  \RequirePackage{xcolor} % [dvipsnames]  
  \definecolor{ingwi}{cmyk}{.9,0,0,0}
\usepackage{textcomp} % fix warning with missing font shapes
\usepackage{scrhack} % fix warnings when using KOMA with listings package          
\usepackage{xspace} % to get the spacing after macros right  
\usepackage{mparhack} % get marginpar right
%\usepackage{fixltx2e} % fixes some LaTeX stuff <-- ist seit 2015 nicht mehr notwendig
\PassOptionsToPackage{}{acronym}
  \usepackage{acronym}
%\renewcommand{\bflabel}[1]{{#1}\hfill} % fix the list of acronyms
\usepackage{booktabs}
\usepackage{multirow}
\usepackage{todonotes} %Settings for ToDoNotes
\usepackage{siunitx}
\usepackage{pgfplots}
\pgfplotsset{compat=1.18}
% Eigene Shortcuts fuer laengere Befehle
  \newcommand{\todox}[1]{\todo[inline, size=\small]{#1}}
  %Nummerierte Anmerkungen
  \newcounter{todocounter}
  \renewcommand{\todox}[2][]{\stepcounter{todocounter}\todo[inline, size=\small,caption={\thetodocounter: #2}, #1]{\renewcommand{\baselinestretch}{0.5}\selectfont\thetodocounter: #2\par}}
\usepackage{blindtext}
%\usepackage{footmisc}
% ****************************************************************************************************


% ****************************************************************************************************
% 3. Setup floats: tables, (sub)figures, and captions
% ****************************************************************************************************
\usepackage{tabularx} % better tables
  \setlength{\extrarowheight}{3pt} % increase table row height
%\newcommand{\myfloatalign}{\centering} % to be used with each float for alignment
\usepackage{caption}
\captionsetup{format=hang,font=small}
\usepackage{subfig}
\usepackage{wrapfig}
% ****************************************************************************************************


% ****************************************************************************************************
% 6. Setup code listings
% ****************************************************************************************************
\usepackage{listings} 
%\lstset{emph={trueIndex,root},emphstyle=\color{BlueViolet}}%\underbar} % for special keywords
\lstset{language=[LaTeX]Tex,%C++,
    keywordstyle=\color{RoyalBlue},%\bfseries,
    basicstyle=\small\ttfamily,
    %identifierstyle=\color{NavyBlue},
    commentstyle=\color{Green}\ttfamily,
    captionpos=b,
    stringstyle=\rmfamily,
    numbers=none,%left,%
    numberstyle=\scriptsize,%\tiny
    stepnumber=5,
    numbersep=8pt,
    showstringspaces=false,
    breaklines=true,
    frameround=ftff,
    frame=single,
    mathescape=false
    texcl=true,
    belowcaptionskip=.75\baselineskip
    %frame=L
} 
%Styles für verschiedene Sprachen festlegen, z.B. Java
\lstdefinestyle{Java}{
belowcaptionskip=1\baselineskip,
  breaklines=true,
  xleftmargin=\parindent,
  language=Java,
  texcl=true,
  showstringspaces=false,
  basicstyle=\footnotesize\ttfamily,
  keywordstyle=\bfseries\color{green!40!black},
  commentstyle=\itshape\color{purple!40!black},
  identifierstyle=\color{blue},
  stringstyle=\color{orange}}
% ****************************************************************************************************    		   


% ****************************************************************************************************
% 6. PDFLaTeX, hyperreferences and citation backreferences
% ****************************************************************************************************
% ********************************************************************
% Using PDFLaTeX
% ********************************************************************
\usepackage{xurl}  % better handling of long URLs / hyphenation
\PassOptionsToPackage{pdftex,hyperfootnotes=false,pdfpagelabels}{hyperref}
  \usepackage{hyperref}  % backref linktocpage pagebackref
\pdfcompresslevel=9
\pdfadjustspacing=1 
\PassOptionsToPackage{pdftex}{graphicx}
  \usepackage{graphicx} 

\usepackage{amssymb} % for \checkmark
\usepackage{xcolor}  % for color definitions
\definecolor{Green}{rgb}{0,0.6,0} % define Green color

% ********************************************************************
% Hyperreferences
% ********************************************************************
\hypersetup{%
    %draft,	% = no hyperlinking at all (useful in b/w printouts)
    pdfstartpage=1, pdfstartview=Fit,%
  colorlinks=true, linktocpage=true,
  %urlcolor=Black, linkcolor=Black, citecolor=Black, %pagecolor=Black,%
  %urlcolor=brown, linkcolor=RoyalBlue, citecolor=green, %pagecolor=RoyalBlue,%
    % uncomment the following line if you want to have black links (e.g., for printing)
    colorlinks=false, pdfborder={0 0 0},
    breaklinks=true, pdfpagemode=UseNone, pageanchor=true, pdfpagemode=UseOutlines,%
    plainpages=false, bookmarksnumbered, bookmarksopen=true, bookmarksopenlevel=1,%
    hypertexnames=true, pdfhighlight=/O,%nesting=true,%frenchlinks,%
    pdftitle={\myTitle},%
    pdfauthor={\textcopyright\ \myName, \myUni},%
    pdfsubject={},%
    pdfkeywords={},%
    pdfcreator={pdfLaTeX},%
    pdfproducer={LaTeX with hyperref}%
}   

% ********************************************************************
% Setup autoreferences
% ********************************************************************
% There are some issues regarding autorefnames
% http://www.ureader.de/msg/136221647.aspx
% http://www.tex.ac.uk/cgi-bin/texfaq2html?label=latexwords
% you have to redefine the makros for the 
% language you use, e.g., american, ngerman
% (as chosen when loading babel/AtBeginDocument)
% ********************************************************************
\makeatletter
\@ifpackageloaded{babel}%
    {%
       \addto\extrasamerican{%
          \renewcommand*{\figureautorefname}{Figure}%
          \renewcommand*{\tableautorefname}{Table}%
          \renewcommand*{\partautorefname}{Part}%
          \renewcommand*{\chapterautorefname}{Chapter}%
          \renewcommand*{\sectionautorefname}{Section}%
          \renewcommand*{\subsectionautorefname}{Section}%
          \renewcommand*{\subsubsectionautorefname}{Section}% 	
        }%
       \addto\extrasngerman{% 
          \renewcommand*{\chapterautorefname}{Kapitel}%
          \renewcommand*{\sectionautorefname}{Abschnitt}%
          \renewcommand*{\subsectionautorefname}{Abschnitt}%
          \renewcommand*{\subsubsectionautorefname}{Abschnitt}% 
          \renewcommand*{\paragraphautorefname}{Absatz}%
          \renewcommand*{\subparagraphautorefname}{Absatz}%
          \renewcommand*{\footnoteautorefname}{Fußnote}%
          \renewcommand*{\FancyVerbLineautorefname}{Zeile}%
          \renewcommand*{\theoremautorefname}{Theorem}%
          \renewcommand*{\appendixautorefname}{Anhang}%
          \renewcommand*{\equationautorefname}{Gleichung}%        
          \renewcommand*{\itemautorefname}{Punkt}%
        }%	
      % Fix to getting autorefs for subfigures right (thanks to Belinda Vogt for changing the definition)
      \providecommand{\subfigureautorefname}{\figureautorefname}%  			
    }{\relax}
\makeatother


% ****************************************************************************************************
% 6. Last calls before the bar closes
% ****************************************************************************************************
% ********************************************************************
% Development Stuff
% ********************************************************************
%\listfiles
\PassOptionsToPackage{l2tabu,orthodox,abort}{nag}
  \usepackage{nag}
%\PassOptionsToPackage{warning, all}{onlyamsmath}
%	\usepackage{onlyamsmath}


% ****************************************************************************************************
% 7. Further adjustments (experimental)
% ****************************************************************************************************
%\usepackage{tocbibind} %Allows us to add Bibliography to ToC
\usepackage{enumitem}
\setdescription{font=\normalfont\bfseries} %Changes the appearance of description items
\usepackage[activate={true,nocompatibility},final,tracking=true,kerning=true,spacing=true,factor=1100,stretch=10,shrink=10]{microtype}


% ********************************************************************
% Using different fonts
% ********************************************************************
%\usepackage{lmodern} % <-- no osf support :-(
%\usepackage[tt=false]{libertine} % [osf]
%\usepackage{luximono} 
%\usepackage[urw-garamond]{mathdesign} <-- no osf support :-(
\usepackage{mathpazo} 
%\usepackage{libertinus} % math? osf?

\setkomafont{disposition}{\bfseries}
% ****************************************************************************************************