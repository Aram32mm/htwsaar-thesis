\chapter{System Design}
\label{ch:system-design}

This chapter presents the architectural design of our hybrid retrieval system for Kotlin validation rules. We detail the monolithic architecture, component interactions, data flow pipelines, and design decisions that enable sub-second retrieval latency while maintaining strict compliance with banking infrastructure constraints. The system operates without external dependencies, implements deterministic retrieval algorithms, and provides comprehensive audit trails for regulatory compliance.

\section{Architectural Overview}

The system implements a monolithic Python architecture optimized for deployment in restricted banking environments. This design choice reflects operational constraints: no external vector databases, no query-time LLM calls, and no distributed dependencies. All components run within a single Dash application process, sharing memory-efficient data structures and eliminating network overhead.

\subsection{Design Principles}

Our architecture adheres to five core principles that guide all implementation decisions:

\begin{enumerate}[leftmargin=*,itemsep=2pt,topsep=2pt]
  \item \textbf{Self-Contained Operation}: Complete functionality without external services, enabling deployment in air-gapped environments
  \item \textbf{Deterministic Retrieval}: Reproducible results for audit compliance, with all randomness eliminated from scoring
  \item \textbf{Memory Efficiency}: In-memory indices for 1,157 rules maintained within modest memory footprint through careful data structure selection
  \item \textbf{Graceful Degradation}: Individual component failures do not cascade; fallback to keyword search if embeddings unavailable
  \item \textbf{Transparent Processing}: Full explainability of ranking decisions through stored score breakdowns
\end{enumerate}

\subsection{Monolithic Architecture}

\begin{figure}[ht]
\centering
\begin{tikzpicture}[
  >=latex,
  node distance=1.5cm,
  component/.style={rectangle, draw, minimum width=3.5cm, minimum height=1cm, align=center},
  ui/.style={component, fill=blue!20},
  engine/.style={component, fill=green!20},
  data/.style={component, fill=orange!20},
  arrow/.style={->, thick}
]

% UI Layer
\node[ui] (search) {Search UI\\{\scriptsize Filters, Results}};
\node[ui, right=of search] (generator) {Generator UI\\{\scriptsize Chat, Upload}};

% RAG Engine Layer
\node[engine, below=2cm of search, xshift=2cm] (rag) {RAG Engine\\{\scriptsize Retriever, Embeddings, Scorer}};

% Data Layer
\node[data, below=of rag] (sqlite) {SQLite Cache\\{\scriptsize Rules, Metadata}};
\node[data, below=of sqlite] (csv) {CSV Corpus\\{\scriptsize 1,157 Rules}};

% Connections
\draw[arrow] (search) -- (rag);
\draw[arrow] (generator) -- (rag);
\draw[arrow, <->] (search) -- (generator);
\draw[arrow] (rag) -- (sqlite);
\draw[arrow] (sqlite) -- (csv);

% Annotations
\node[anchor=west] at (6,0) {\textbf{Presentation}};
\node[anchor=west] at (6,-2) {\textbf{Business Logic}};
\node[anchor=west] at (6,-4.5) {\textbf{Data Access}};
\node[anchor=west] at (6,-6) {\textbf{Storage}};

% Process boundary
\draw[dashed, thick, rounded corners] (-2,1) rectangle (5,-7) 
  node[pos=0, anchor=north west] {\small Single Dash Process};

\end{tikzpicture}
\caption{Monolithic architecture with all components running in a single Dash application process. The UI layer provides search and generator interfaces, the RAG engine handles retrieval and scoring, and the data layer manages persistent storage and caching.}
\label{fig:monolithic-architecture}
\end{figure}

The monolithic design offers several advantages for our use case:

\begin{itemize}[leftmargin=*,itemsep=2pt,topsep=2pt]
  \item \textbf{Simplified Deployment}: Single Python application with standard library dependencies
  \item \textbf{Shared Memory}: Indices and embeddings loaded once, accessible to all components
  \item \textbf{No Network Latency}: All communication happens through function calls
  \item \textbf{Atomic Updates}: Corpus changes require simple application restart
  \item \textbf{Audit Compliance}: Single process simplifies logging and monitoring
\end{itemize}

\section{Data Flow Architecture}

\subsection{Startup Initialization}

The system follows a deterministic initialization sequence at application startup:

\begin{figure}[ht]
\centering
\begin{tikzpicture}[
  >=latex,
  node distance=1.2cm,
  process/.style={rectangle, draw, minimum width=4cm, minimum height=0.8cm, align=center},
  data/.style={cylinder, draw, shape border rotate=90, minimum width=2cm, minimum height=1cm, align=center},
  arrow/.style={->, thick}
]

% Startup sequence
\node[data] (csv) {CSV Corpus};
\node[process, below=of csv] (parse) {Parse \& Validate};
\node[process, below=of parse] (sqlite) {Load SQLite};
\node[process, below=of sqlite] (embeddings) {Parse Embeddings\\{\scriptsize JSON → numpy}};
\node[process, below=of embeddings] (normalize) {L2 Normalize};
\node[process, below=of normalize] (indices) {Build Indices\\{\scriptsize BM25, Fuzzy, Semantic}};
\node[process, below=of indices] (cache) {Cache Metadata};
\node[process, below=of cache] (ready) {System Ready};

% Flow
\draw[arrow] (csv) -- (parse);
\draw[arrow] (parse) -- (sqlite);
\draw[arrow] (sqlite) -- (embeddings);
\draw[arrow] (embeddings) -- (normalize);
\draw[arrow] (normalize) -- (indices);
\draw[arrow] (indices) -- (cache);
\draw[arrow] (cache) -- (ready);

\end{tikzpicture}
\caption{Startup initialization sequence. The process loads the CSV, validates data integrity, parses embeddings from JSON, builds all search indices, and caches metadata for fast access.}
\label{fig:startup-sequence}
\end{figure}

Key initialization steps:
\begin{itemize}[leftmargin=*,itemsep=2pt,topsep=2pt]
  \item \textbf{CSV Loading}: Read and validate the standardized corpus
  \item \textbf{Embedding Parsing}: Convert JSON strings to numpy arrays
  \item \textbf{Index Building}: Construct BM25, fuzzy, and semantic indices in memory
  \item \textbf{Metadata Caching}: Pre-compute filter options and statistics
\end{itemize}

\subsection{Query-Time Data Flow}

Once initialized, the system processes queries through a multi-stage pipeline:

\begin{figure}[ht]
\centering
\begin{tikzpicture}[
  >=latex,
  thick,
  node distance=1cm,
  process/.style={rectangle, draw, minimum width=3cm, minimum height=0.8cm, align=center, font=\small},
  parallel/.style={process, fill=yellow!20},
  merge/.style={process, fill=green!20},
  arrow/.style={->, thick}
]

% Query input
\node[process] (query) {User Query};

% Filter application
\node[process, below=of query] (filter) {Apply Filters\\{\scriptsize Rule Type, Country, etc.}};

% Parallel retrieval
\node[parallel, below left=1.5cm and 2cm of filter] (bm25) {BM25\\{\scriptsize Keywords}};
\node[parallel, below=1.5cm of filter] (fuzzy) {Fuzzy\\{\scriptsize Rule Names}};
\node[parallel, below right=1.5cm and 2cm of filter] (semantic) {Semantic\\{\scriptsize Embeddings}};

% Score normalization
\node[process, below=1.5cm of fuzzy] (normalize) {Per-Signal\\Normalization};

% Hybrid scoring
\node[merge, below=of normalize] (hybrid) {Hybrid Scoring\\{\scriptsize w=[0.10, 0.10, 0.80]}};

% Semantic gate
\node[process, below=of hybrid] (gate) {Semantic Gate\\{\scriptsize $\tau$=0.30}};

% Ranking
\node[process, below=of gate] (rank) {Rank \& Select\\{\scriptsize Top-k=10}};

% Results
\node[process, below=of rank] (results) {Format Results};

% Connections
\draw[arrow] (query) -- (filter);
\draw[arrow] (filter) -- (bm25);
\draw[arrow] (filter) -- (fuzzy);
\draw[arrow] (semantic) |- (normalize);
\draw[arrow] (bm25) |- (normalize);
\draw[arrow] (fuzzy) -- (normalize);
\draw[arrow] (normalize) -- (hybrid);
\draw[arrow] (hybrid) -- (gate);
\draw[arrow] (gate) -- (rank);
\draw[arrow] (rank) -- (results);

\end{tikzpicture}
\caption{Query-time data flow with parallel retrieval and sequential scoring. The three retrieval signals operate independently before normalization and combination.}
\label{fig:query-flow}
\end{figure}

\section{Component Design}

\subsection{Retrieval Components}

The system implements three complementary retrieval signals, each targeting different aspects of the search space:

\subsubsection{BM25 Component}
Provides lexical matching over the curated keywords field using standard BM25 scoring with term frequency saturation and document length normalization. The implementation uses the \texttt{rank-bm25} library with default parameters.

\subsubsection{Fuzzy Matching Component}
Handles variations in rule naming through token-set ratio matching, which is order-independent and robust to minor spelling variations. This component catches rules that users might remember imprecisely.

\subsubsection{Semantic Component}
Dense retrieval using cosine similarity over L2-normalized embeddings. Since all embeddings are pre-normalized at startup, cosine similarity reduces to a simple dot product, enabling efficient matrix multiplication.

\subsection{Hybrid Scoring Design}

The hybrid scorer combines signals through a three-stage process:

\subsubsection{Per-Signal Normalization}
Each signal's scores are normalized independently to $[0,1]$ using min-max normalization:

\begin{align}
\hat{s}_i &= \frac{s_i - \min_j(s_j)}{\max_j(s_j) - \min_j(s_j)}
\end{align}

This ensures fair combination despite different score distributions across signals.

\subsubsection{Semantic Gating}
Low-relevance semantic matches are filtered to reduce noise:

\begin{align}
s'_{\text{sem}} &= \begin{cases}
  \hat{s}_{\text{sem}} & \text{if } \hat{s}_{\text{sem}} \geq \tau \\
  0 & \text{otherwise}
\end{cases}
\end{align}

with $\tau = 0.30$ determined empirically through evaluation.

\subsubsection{Weighted Combination}
Final scores use LOOCV-optimized weights:

\begin{align}
s_{\text{hybrid}} &= 0.10 \cdot \hat{s}_{\text{bm25}} + 0.10 \cdot \hat{s}_{\text{fuzzy}} + 0.80 \cdot s'_{\text{sem}}
\end{align}

\section{User Interface Design}

\subsection{Dash Application Architecture}

The UI layer implements a reactive single-page application using Dash:

\begin{table}[ht]
\centering
\begin{tabular}{ll}
\toprule
\textbf{Component} & \textbf{Function} \\
\midrule
Search Bar & Natural language query input with debouncing \\
Mode Selector & Toggle between Keyword and Hybrid modes \\
Facet Filters & Dropdowns for Rule Type, Country, Business Type, Party Agent \\
Results Panel & Ranked cards with badges, scores, and code display \\
Explanation Panel & Grounded explanations from stored fields \\
Metrics Footer & Query latency and result count display \\
\bottomrule
\end{tabular}
\caption{UI components in the Dash application providing comprehensive search functionality.}
\label{tab:ui-components}
\end{table}

\subsection{Callback Architecture}

Dash callbacks implement the reactive data flow between UI components and the retrieval engine. User interactions trigger callbacks that invoke the retriever, process results, and update the display. The architecture maintains clear separation between presentation logic and retrieval logic.

\section{Data Management Design}

\subsection{CSV-First Architecture}

The CSV corpus serves as the single source of truth with several advantages:

\begin{itemize}[leftmargin=*,itemsep=2pt,topsep=2pt]
  \item \textbf{Simplicity}: Standard format readable by any tool
  \item \textbf{Version Control}: Git-trackable with meaningful diffs
  \item \textbf{Audit Trail}: Complete history of all changes
  \item \textbf{Portability}: No database migration required
  \item \textbf{Embedding Storage}: JSON strings preserve full precision
\end{itemize}

\subsection{SQLite Integration}

SQLite provides lightweight data management for filtering and metadata:

\begin{itemize}[leftmargin=*,itemsep=2pt,topsep=2pt]
  \item \textbf{Filtering}: Efficient WHERE clauses for faceted search
  \item \textbf{Caching}: Store pre-computed filter combinations
  \item \textbf{Transactions}: ACID compliance for concurrent reads
  \item \textbf{Memory Mode}: Optional in-memory database for speed
\end{itemize}

Note that SQLite is used only for filtering and metadata management. The actual search indices (BM25, fuzzy, semantic) are built and maintained entirely in memory at startup.

\section{Performance Optimization}

\subsection{Memory Management}

The system carefully manages memory through several strategies:

\begin{itemize}[leftmargin=*,itemsep=2pt,topsep=2pt]
  \item \textbf{Shared Arrays}: NumPy arrays shared across components without copying
  \item \textbf{Float32 Precision}: Embeddings stored as float32 instead of float64
  \item \textbf{Lazy Loading}: Parse embeddings only when first accessed
  \item \textbf{Result Limiting}: Cap candidate pools to prevent memory spikes
  \item \textbf{Garbage Collection}: Explicit cleanup after batch operations
\end{itemize}

\subsection{Latency Optimization}

Query latency is minimized through:

\begin{itemize}[leftmargin=*,itemsep=2pt,topsep=2pt]
  \item \textbf{Pre-computed Indices}: All indices built at startup, not query time
  \item \textbf{Vectorized Operations}: NumPy operations instead of Python loops
  \item \textbf{Matrix Multiplication}: Batch similarity computation
  \item \textbf{Partial Sorting}: Only sort top-k results, not entire list
  \item \textbf{Component Memoization}: Cache UI components that don't change
\end{itemize}

\section{Design Trade-offs}

\subsection{Monolithic vs. Microservices}

We chose monolithic architecture despite microservices being the modern standard:

\textbf{Advantages of our monolithic approach:}
\begin{itemize}[leftmargin=*,itemsep=2pt,topsep=2pt]
  \item Single deployment unit simplifies operations
  \item No network overhead between components  
  \item Shared memory reduces duplication
  \item Easier debugging and profiling
  \item Guaranteed consistency without distributed transactions
\end{itemize}

\textbf{Accepted limitations:}
\begin{itemize}[leftmargin=*,itemsep=2pt,topsep=2pt]
  \item Vertical scaling only (sufficient for our corpus size)
  \item Full restart required for updates
  \item Single point of failure (mitigated by process supervision)
  \item Language lock-in to Python
\end{itemize}

\subsection{CSV vs. Database Storage}

Storing embeddings in CSV as JSON strings is unconventional but practical:

\textbf{Benefits:}
\begin{itemize}[leftmargin=*,itemsep=2pt,topsep=2pt]
  \item Single file contains entire corpus
  \item Human-readable for debugging
  \item Version control tracks all changes
  \item No database administration overhead
\end{itemize}

\textbf{Trade-offs:}
\begin{itemize}[leftmargin=*,itemsep=2pt,topsep=2pt]
  \item Larger file size than binary formats
  \item Parsing overhead at startup
  \item No incremental updates
\end{itemize}

\subsection{In-Memory Indices vs. Persistent Indices}

Building indices at startup ensures consistency:

\textbf{Advantages:}
\begin{itemize}[leftmargin=*,itemsep=2pt,topsep=2pt]
  \item Always synchronized with current data
  \item No index corruption issues
  \item Simplified deployment (no index files)
  \item Predictable memory usage
\end{itemize}

\textbf{Cost:}
\begin{itemize}[leftmargin=*,itemsep=2pt,topsep=2pt]
  \item Startup latency (acceptable for our use case)
  \item Memory usage for index structures
\end{itemize}

\section{Security and Compliance}

\subsection{Security Measures}

The monolithic design simplifies security:

\begin{itemize}[leftmargin=*,itemsep=2pt,topsep=2pt]
  \item \textbf{Input Validation}: Query sanitization prevents injection
  \item \textbf{Read-Only Operations}: No data modification endpoints
  \item \textbf{Process Isolation}: Runs as non-privileged user
  \item \textbf{No External Calls}: Eliminates API key management
  \item \textbf{Audit Logging}: All queries logged with timestamps
\end{itemize}

\subsection{Compliance Features}

Banking requirements are addressed through:

\begin{itemize}[leftmargin=*,itemsep=2pt,topsep=2pt]
  \item \textbf{Reproducibility}: Deterministic scoring algorithms
  \item \textbf{Explainability}: Score breakdowns stored and displayed
  \item \textbf{Data Lineage}: CSV versioning tracks all changes
  \item \textbf{Access Control}: Integration with enterprise SSO
  \item \textbf{Data Residency}: All processing happens on-premises
\end{itemize}

\section{Summary}

The system design prioritizes simplicity, reliability, and compliance over architectural elegance. The monolithic Dash application with CSV storage and SQLite caching may seem anachronistic compared to modern microservices with vector databases, but it precisely fits our operational constraints. By building indices at startup, normalizing scores per query, and carefully tuning the hybrid weights, we achieve competitive retrieval performance with excellent latency characteristics.

The key insight is that sophisticated NLP capabilities---semantic search, hybrid scoring, faceted filtering---can be delivered through conventional architectures when thoughtfully designed. Our system proves that banking-compliant information retrieval does not require complex infrastructure; it requires careful engineering within constraints.
