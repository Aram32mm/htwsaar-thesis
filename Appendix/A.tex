%********************************************************************
% Appendix A – System Configuration and Implementation Details
%*******************************************************
\chapter{System Configuration and Implementation Details}
\label{app:implementation}

This appendix provides comprehensive technical details about the hybrid retrieval system implementation, including configuration parameters, API specifications, deployment instructions, and performance measurements.

\section{System Configuration}
\label{sec:app-config}

\subsection{Search Configuration Parameters}

The system's retrieval behavior is controlled through configuration parameters tuned via Leave-One-Out Cross-Validation:

\begin{table}[!htbp]
\centering
\begin{tabular}{lll}
\toprule
\textbf{Parameter} & \textbf{Value} & \textbf{Description} \\
\midrule
\texttt{SEMANTIC\_WEIGHT} & 0.80 & Weight for semantic similarity signal \\
\texttt{BM25\_WEIGHT} & 0.10 & Weight for BM25 keyword matching \\
\texttt{FUZZY\_WEIGHT} & 0.10 & Weight for fuzzy string matching \\
\texttt{MIN\_SIMILARITY} & 0.30 & Semantic gate threshold ($\tau$) \\
\texttt{ENABLE\_RERANKING} & False & Metadata-based reranking flag \\
\texttt{DEFAULT\_SEARCH\_LIMIT} & 10 & Maximum results per query \\
\texttt{EMBEDDING\_MODEL} & UAE-Large-V1 & Transformer model path \\
\texttt{EMBEDDING\_DIM} & 1024 & Embedding vector dimensions \\
\bottomrule
\end{tabular}
\caption{Core search configuration parameters with LOOCV-tuned values.}
\label{tab:config-params}
\end{table}

\subsection{Database Configuration}

SQLite database settings optimized for concurrent read access:

\begin{lstlisting}[language=Python, caption={SQLite database configuration}, label={lst:db-config}]
# Database paths
SQLITE_DB_PATH = "db/rules.db"
SQLITE_TABLE_NAME = "rules"
CSV_DATA_PATH = "data/validation_rules.csv"

# SQLite pragmas for optimization
PRAGMA journal_mode = WAL;      # Write-Ahead Logging
PRAGMA synchronous = NORMAL;    # Balance safety/speed
PRAGMA foreign_keys = ON;       # Referential integrity
PRAGMA busy_timeout = 5000;     # 5 second timeout
PRAGMA cache_size = -64000;     # 64MB page cache
\end{lstlisting}

\subsection{Application Settings}

Dash web application configuration:

\begin{lstlisting}[language=Python, caption={Application configuration}, label={lst:app-settings}]
# Server settings
APP_TITLE = "Validation Rules Search Engine"
DEBUG = 1                        # Development mode
HOST = "127.0.0.1"              # Bind address
PORT = 8050                     # Application port

# Feature flags
CHAT_ENABLED = 1                # Enable chat interface
UPDATE_EMBEDDINGS = 1           # Recompute on startup

# Generator settings
GENERATOR_MESSAGE_WINDOW = 10   # Chat history size
GENERATOR_TEMPERATURE = 0.7     # LLM temperature
GENERATOR_MAX_TOKENS = 512      # Response length limit
\end{lstlisting}

\clearpage
\section{Data Schema}
\label{sec:app-schema}

\subsection{Input CSV Schema}

The standardized CSV corpus contains only essential rule data:

\begin{table}[!htbp]
\centering
\small
\begin{tabular}{p{3.5cm}p{2cm}p{6cm}}
\toprule
\textbf{Field} & \textbf{Type} & \textbf{Description} \\
\midrule
\texttt{rule\_id} & TEXT & Unique identifier \\
\texttt{rule\_name} & TEXT & Human-readable name \\
\texttt{rule\_description} & TEXT & Original business description \\
\texttt{bansta\_error\_code} & TEXT & Banking error code \\
\texttt{iso\_error\_code} & TEXT & ISO 20022 error code \\
\texttt{description\_en} & TEXT & English description \\
\texttt{description\_de} & TEXT & German description \\
\texttt{rule\_code} & TEXT & Kotlin implementation \\
\texttt{llm\_description} & TEXT & LLM-enhanced description \\
\texttt{keywords} & TEXT & Comma-separated keywords \\
\texttt{rule\_type} & TEXT & Comma-separated types \\
\texttt{country} & TEXT & Comma-separated countries \\
\texttt{business\_type} & TEXT & Comma-separated business types \\
\texttt{party\_agent} & TEXT & Comma-separated agents \\
\texttt{embedding} & TEXT & JSON array of 1024 floats \\
\bottomrule
\end{tabular}
\caption{Input CSV schema containing rule-specific data.}
\label{tab:csv-schema}
\end{table}

\subsection{Database Metadata Fields}

Additional metadata fields are automatically added during database ingestion:

\begin{table}[!htbp]
\centering
\small
\begin{tabular}{p{3.5cm}p{2cm}p{6cm}}
\toprule
\textbf{Field} & \textbf{Type} & \textbf{Default Value} \\
\midrule
\texttt{relevance} & REAL & 1.0 (maximum relevance) \\
\texttt{version\_major} & INTEGER & 1 \\
\texttt{version\_minor} & INTEGER & 0 \\
\texttt{created\_at} & TEXT & Current ISO 8601 timestamp \\
\texttt{updated\_at} & TEXT & Current ISO 8601 timestamp \\
\bottomrule
\end{tabular}
\caption{Metadata fields automatically added during database ingestion.}
\label{tab:metadata-fields}
\end{table}

\subsection{Embedding Storage Format}

Embeddings are serialized as JSON arrays for portability and compatibility:

\begin{lstlisting}[language=Python, caption={Embedding serialization and deserialization}, label={lst:embedding-format}]
# During corpus preparation (offline)
import json
import numpy as np

embedding_vector = model.encode(text)  # Shape: (1024,)
embedding_json = json.dumps(embedding_vector.tolist())
# Store in CSV as string

# At runtime (online)
embedding_array = np.array(
   json.loads(embedding_json), 
   dtype=np.float32  # Memory optimization
)
# L2 normalize for cosine similarity
norm = np.linalg.norm(embedding_array) + 1e-12
embedding_normalized = embedding_array / norm
\end{lstlisting}

\clearpage
\section{API Specifications}
\label{sec:app-api}

\subsection{RuleRetriever Interface}

The main retrieval class exposes three public methods for search and exploration:

\begin{lstlisting}[language=Python, caption={RuleRetriever public API}, label={lst:retriever-api}]
class RuleRetriever:
   def search_rules(
       self,
       query: Optional[str] = None,
       rule_type: Optional[List[str]] = None,
       country: Optional[List[str]] = None,
       business_type: Optional[List[str]] = None,
       party_agent: Optional[List[str]] = None,
       mode: SearchMode = SearchMode.HYBRID,
       top_k: int = 10
   ) -> List[dict]:
       """
       Search for validation rules.
       Returns list of rules with search_score field.
       """
       
   def get_filter_options(self) -> Dict[str, List[str]]:
       """
       Get available filter values for UI dropdowns.
       Returns dict mapping field names to unique values.
       """
       
   def candidate_pool(
       self, 
       query: str, 
       k_each: int = 20
   ) -> List[dict]:
       """
       Get union of top-k from each signal.
       Used for evaluation experiments only.
       """
\end{lstlisting}

\subsection{Search Modes}

The system supports four distinct retrieval modes:

\begin{lstlisting}[language=Python, caption={Search mode enumeration}, label={lst:search-modes}]
from enum import Enum

class SearchMode(Enum):
   HYBRID = "hybrid"      # Weighted combination (default)
   SEMANTIC = "semantic"  # Embedding similarity only
   KEYWORD = "keyword"    # BM25 only
   FUZZY = "fuzzy"       # String similarity only
\end{lstlisting}

\clearpage
\section{Deployment Instructions}
\label{sec:app-deployment}

\subsection{Local Development Setup}

\begin{lstlisting}[language=bash, caption={Local development setup}, label={lst:local-setup}]
# Clone repository
git clone <repository-url>
cd validation-rule-search

# Create virtual environment
python -m venv env
source env/bin/activate  # Windows: env\Scripts\activate

# Install dependencies
pip install -r requirements.txt

# Run application
python app.py

# Access at http://localhost:8050
\end{lstlisting}

\subsection{Production Deployment}

\begin{lstlisting}[language=bash, caption={Production deployment with gunicorn}, label={lst:prod-deploy}]
# Install production server
pip install gunicorn

# Run with multiple workers
gunicorn app:server \
   --workers 4 \
   --worker-class sync \
   --bind 0.0.0.0:8050 \
   --timeout 120 \
   --log-level info
\end{lstlisting}

\subsection{Systemd Service Configuration}

\begin{lstlisting}[language=ini, caption={Systemd service configuration}, label={lst:systemd}]
[Unit]
Description=Validation Rule Search Engine
After=network.target

[Service]
Type=simple
User=appuser
WorkingDirectory=/opt/validation-rule-search
Environment="PATH=/opt/validation-rule-search/env/bin"
ExecStart=/opt/validation-rule-search/env/bin/gunicorn \
   app:server --workers 4 --bind 0.0.0.0:8050
Restart=always
RestartSec=10

[Install]
WantedBy=multi-user.target
\end{lstlisting}

\clearpage
\section{Performance Measurements}
\label{sec:app-performance}

\subsection{Test Environment}

Performance measurements were conducted on a development machine with the following specifications:
\begin{itemize}[leftmargin=*,itemsep=2pt,topsep=2pt]
 \item Python 3.x runtime environment
 \item Full corpus of 1,157 validation rules
 \item 50 diverse test queries spanning different types and lengths
 \item All measurements averaged over multiple runs
\end{itemize}

\subsection{System Performance Metrics}

\begin{table}[!htbp]
\centering
\begin{tabular}{lr}
\toprule
\textbf{Metric} & \textbf{Measured Value} \\
\midrule
Startup time & 464 ms \\
Index build time & 72 ms \\
Memory usage & 530 MB \\
Memory percentage & 3.23\% \\
Database full read (mean) & 27 ms \\
Database full read (P95) & 32 ms \\
Total rules indexed & 1,157 \\
\bottomrule
\end{tabular}
\caption{System initialization and resource metrics.}
\label{tab:system-metrics}
\end{table}

\subsection{Search Latency by Mode}

\begin{table}[!htbp]
\centering
\begin{tabular}{lrrrr}
\toprule
\textbf{Search Mode} & \textbf{Mean} & \textbf{Median} & \textbf{P95} & \textbf{P99} \\
& \textbf{(ms)} & \textbf{(ms)} & \textbf{(ms)} & \textbf{(ms)} \\
\midrule
Keyword (BM25) & 1.2 & 1.1 & 2.2 & 2.4 \\
Semantic & 2.4 & 2.4 & 3.1 & 3.6 \\
Fuzzy & 19.4 & 18.3 & 35.8 & 36.3 \\
Hybrid & 189.3 & 58.1 & 659.8 & 2129.4 \\
\bottomrule
\end{tabular}
\caption{Query latency distribution across different search modes.}
\label{tab:search-latency}
\end{table}

\subsection{Filter Impact on Performance}

\begin{table}[!htbp]
\centering
\begin{tabular}{lrr}
\toprule
\textbf{Filter Configuration} & \textbf{Mean (ms)} & \textbf{P95 (ms)} \\
\midrule
No filters (full corpus) & 35.7 & 36.4 \\
Single filter applied & 1.2 & 1.5 \\
Multiple filters applied & 0.3 & 0.4 \\
\bottomrule
\end{tabular}
\caption{Performance improvement with filter application.}
\label{tab:filter-impact}
\end{table}

\clearpage
\section{Data Preparation Pipeline}
\label{sec:app-pipeline}

\subsection{Standardization Process}

The validation rule corpus undergoes a seven-stage standardization pipeline:

\begin{enumerate}[leftmargin=*,itemsep=3pt,topsep=3pt]
 \item \textbf{Collection}: Aggregate rules from distributed sources
 \item \textbf{Deduplication}: Remove duplicate rules by \texttt{rule\_id}
 \item \textbf{Field Mapping}: Normalize column names and formats
 \item \textbf{LLM Enhancement}: Generate descriptions using Gemini-2.5-Pro
 \item \textbf{Keyword Extraction}: Extract search terms from descriptions
 \item \textbf{Embedding Generation}: Compute UAE-Large-V1 vectors
 \item \textbf{Export}: Save as standardized CSV format
\end{enumerate}

\subsection{Embedding Generation Script}

The following script generates embeddings for all rules in batch:

\begin{lstlisting}[language=Python, caption={Batch embedding generation script}, label={lst:batch-embed}]
def generate_embeddings(csv_path, model_name="UAE-Large-V1"):
   # Load pre-trained model
   tokenizer = AutoTokenizer.from_pretrained(model_name)
   model = AutoModel.from_pretrained(model_name)
   
   # Load rule data
   df = pd.read_csv(csv_path)
   embeddings = []
   
   # Process each rule description
   for text in df['llm_description']:
       inputs = tokenizer(
           text, 
           return_tensors='pt',
           truncation=True, 
           max_length=512
       )
       
       with torch.no_grad():
           outputs = model(**inputs)
           # Mean pooling across tokens
           emb = outputs.last_hidden_state.mean(dim=1)
           # L2 normalize for cosine similarity
           emb = F.normalize(emb, p=2, dim=1)
           embeddings.append(emb.squeeze().tolist())
   
   # Save embeddings as JSON strings
   df['embedding'] = [json.dumps(e) for e in embeddings]
   df.to_csv(csv_path, index=False)
   print(f"Generated embeddings for {len(df)} rules")
\end{lstlisting}

\section{Summary}

This appendix provides the complete technical specification for the hybrid retrieval system. The configuration parameters enable fine-tuning for different deployments, while the API specifications ensure consistent integration. Performance measurements demonstrate production readiness with sub-second response times for most queries. The data preparation pipeline ensures reproducible corpus standardization, enabling other organizations to implement similar systems with their own validation rule sets.
