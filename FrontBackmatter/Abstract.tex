%*******************************************************
% Abstract
%*******************************************************
\pdfbookmark[0]{Abstract}{abs}
\chapter*{Abstract}
\markboth{Abstract}{Abstract}
\addcontentsline{toc}{chapter}{Abstract}

Financial institutions maintain thousands of payment validation rules with inconsistent documentation across fragmented systems, hindering efficient rule discovery. This thesis presents a retrieval-augmented generation approach where large language models enrich source data offline, enabling semantic search without runtime model dependencies. During corpus preparation, LLMs generate standardized descriptions, extract keywords, and infer categorical metadata from unstructured rule documentation. This enriched data supports a hybrid retrieval system combining sparse retrieval, dense embeddings, and fuzzy matching through weighted aggregation. The monolithic architecture ensures deterministic, auditable retrieval—essential for regulatory compliance—while eliminating external service dependencies. Evaluation through cross-validation demonstrates the hybrid approach outperforms individual retrieval methods, achieving effective rule discovery with sub-second response times. This work shows that RAG benefits can be captured through offline preprocessing rather than online inference, providing a practical framework for deploying advanced NLP capabilities in constrained enterprise environments. The approach bridges modern language understanding with regulatory requirements, offering a template for similar applications in regulated industries.

\medskip
\noindent\textbf{Keywords:} retrieval-augmented generation, offline LLM enrichment, hybrid retrieval, payment validation rules, regulatory compliance, enterprise constraints