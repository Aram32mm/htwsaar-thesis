%*******************************************************
% Abstract
%*******************************************************
\pdfbookmark[0]{Abstract}{abs}
\chapter*{Abstract}
\markboth{Abstract}{Abstract}
\addcontentsline{toc}{chapter}{Abstract}

Financial institutions depend on large sets of validation rules to qualify, populate, and validate payments across networks such as SWIFT and ISO-based rails. Locating the right rule quickly is difficult because descriptions, tags, and naming are inconsistent and dispersed across multiple systems. This thesis presents a production-ready, monolithic retrieval system that retrieves and explains payment validation rules from a standardized CSV corpus—one row per rule—consolidated from distributed sources and enriched with identifiers, bilingual descriptions (EN/DE), BANSTA/ISO error codes, extracted keywords, generated summaries, 1024-dimensional embeddings, scraped categorical tags (Rule Type, Country, Business Type, Party Agent), and relevance metadata.

The approach implements a pragmatic variant of Retrieval-Augmented Generation (RAG) optimized for regulatory compliance. All LLM processing occurs offline during corpus preparation using Gemini-2.5-Pro: generating enhanced rule descriptions, extracting curated keywords, and scraping categorical tags from unstructured text. UAE-Large-V1 produces 1024-dimensional mean-pooled embeddings that are L2-normalized and stored as JSON strings. At startup, the system ingests the CSV into SQLite, builds in-memory indices, and caches filter options. Online retrieval uses only deterministic, auditable components: BM25 over extracted keywords via \texttt{rank\_bm25}; token-set fuzzy matching on rule names via \texttt{fuzzywuzzy}; and cosine similarity via \texttt{scikit-learn}'s \texttt{cosine\_similarity} on the embedding matrix (i.e., normalized dot product). Raw scores undergo per-signal min–max normalization to [0,1], then combine via empirically tuned weights—0.80 (semantic), 0.10 (BM25), and 0.10 (fuzzy)—with a semantic gate at 0.30 that zeros out off-topic results.

The monolithic Dash application provides a complete user interface with faceted filtering (using cached filter options), mode selection (Keyword vs.\ Hybrid), and ranked result cards that display full rule information together with categorical tag badges and kotlin code. The infrastructure includes provisions for a chatbot interface and metadata-based re-ranking (factor 0.10), though these features remain unimplemented in the current version. This architecture was designed for strict banking environments—no FAISS, no external vector databases, no query-time LLM calls—while maintaining P95 latency under 1000\,ms for top-10 retrieval on a corpus of approximately 1000 rules.

Evaluation via Leave-One-Out Cross-Validation (LOOCV) on 30 labeled prompts with ground-truth rule mappings demonstrates that the tuned hybrid approach achieves 48.21\% MRR@5, outperforming individual signals (BM25: 38.78\%, Semantic: 44.04\%, Fuzzy: 24.17\%). Ablation studies confirm each signal's contribution, while sensitivity analysis shows robustness to weight perturbations. The system has been developed for Deutsche Bank's eBridge-EU platform to support quality assurance teams and developers in rule discovery tasks. Limitations include dependency on LLM extraction quality, limited evaluation dataset size, and vertical scaling constraints inherent to monolithic architecture. Future work includes implementing the chatbot interface for conversational rule discovery, deploying metadata-based re-ranking, expanding the evaluation dataset to enable more robust optimization, and preparing the system for production deployment.

\medskip
\noindent\textbf{Keywords:} information retrieval, hybrid search, semantic embeddings, BM25, fuzzy matching, dense retrieval, UAE-Large-V1, mean pooling, L2 normalization, cosine similarity, scikit-learn, SQLite, Dash web framework, Leave-One-Out Cross-Validation, Mean Reciprocal Rank, offline LLM processing, retrieval-augmented generation, faceted search, monolithic architecture, rule-based systems